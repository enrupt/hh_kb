\documentclass[twocolumn]{article}
\title{Про влияние числа откликов на вероятность успешного отклика в HH}

% Language setting
% Replace `english' with e.g. `spanish' to change the document language
\usepackage[utf8]{inputenc}
\usepackage[T1]{fontenc}
\usepackage[russian]{babel}

\usepackage{amssymb,amsmath,bm}
\usepackage{cmap}
\usepackage{amsmath}

\begin{document}
    \date{}
    \maketitle

    \section{Постановка задачи}

    Предположим, есть некоторое множестов кандидатов ${c_i} \in C, i \in \overline{1, N}$ и некоторое число предложений о работе ${w_j} \in W, j \in \overline{1, M}$, каждый из $c_i$ кандидатов откликается на некоторое подмножество предложений ${Re_{i}} \subseteq W \hspace{2mm}  \forall{i} \hspace{1mm} \overline{1, N}$. Также допустим, что у нас уже задана некоторая функция $f(c_i, w_j) \xrightarrow{} [0;1]$, которая с удовлетворительной точностью умеет отвечать на вопрос "подходит ли кандидат $c_i$ на позицию $w_j$.

    Работодатель, рассматривая каждый из откликов может либо принять отклик, либо отклонить его, при этом будем считать, что непросмотренный отклик экививалентен отклоненному. Допустим, с некоторой вероятностью $a_{j}$ работодатель вакансии ${w_j}$ принимает подхоящего кандидата и с какой-то вероятностью $b_{j}$ принимает неподходящего. В общем случае $a_{j}$ и $b_{j}$ могут зависеть от и от самого кандидата ${c_i}$ (а точнее, от $f(c_i, w_j)$), но пока для упрощения рассуждений не будем рассматривать этот случай, тогда вероятность того, что отклик кандидата будет принят выражается следуюшим образом:
    \begin{equation}
        \label{P_def}
        P_{conf}(c_i, w_j) = a_{j} * f(c_i, w_j) + b_{j} * (1 - f(c_i, w_j))
    \end{equation}

    В принятых обозначениях у нас стоит задача максимизация числа принятых откликов. Полагая $b_{j}$ некоторой константой, не зависящей ни от кандидата, ни от числа откликов, будем считать, что наша задача сводится к максимизации следующего выигрыша

    $$
    S = \sum_{i}^{N}\sum_{w_j \in Re_{i}} a_{j}  * f(c_i, w_j) \xrightarrow{} max
    $$

    \section{Идея решения}

    Для иллюстрации идеи рассмотрим пример: имеемтся 2 множества кандидатов $A$ и $B$: $|A| = 100, |B| = 100, X = A \cap B, |X| = 50$, а также 2 вакансии $w_1$ и $w_2$. Предположим, что
    \begin{equation}
        \label{intro_example}
        f(c'_{i}, w_1) = 0 \hspace{2mm}  \forall{c'_i} \in B/X,
        f(c''_{i}, w_2) = 0 \hspace{2mm}  \forall{c''_i} \in A/X
    \end{equation}

    а также, что каждому кандидату из $X$  можем показать только 1 из вакансий, тогда по построению множеств существует 2 возможных значения суммарного выигрыша

    \begin{multline*}
        S_1 = \sum_{c_i \in A / X} a_1 * f(c_i, w_1) + \sum_{c_i \in X} a_1 * f(c_i, w_1) + \\
        + \sum_{c_i \in B / X} a_1 * f(c_i, w_2)
    \end{multline*}

    \begin{multline*}
        S_2 = \sum_{c_i \in A / X} a_1 * f(c_i, w_1) + \sum_{c_i \in X} a_2 * f(c_i, w_2) + \\
        + \sum_{c_i \in B / X} a_1 * f(c_i, w_2)
    \end{multline*}

    Отсюда следует довольно очевидный вывод, что выигрышная стратегия определяется взвешенной суммой функции $f$ на кандидатах, заинтересованных в обеих вакансиях.

    Рассмотрим теперь, как ведет себя целевая функция, когда для некоторого кандидата $c_k$ мы принимаем решение о показе рекомендации $w_r$ для позиции $p + 1$. Не нарушая общности, положим, что были показаны вакансии с 1 по $p$-тую, тогда текущее значение функции выигрыша для всех кандидатов до $k$ включительно равно:

    $$
    S = \sum_{i}^{k-1}\sum_{w_j \in Re_{i}} a_{j}  * f(c_i, w_j) + \sum_{s = 1}^{p} a_{s}  * f(c_k, w_s)
    $$

    Далее нужно заметить, что введенная нами выше вероятность $a_j$ в общем случае не является константой. В частности, справедливым кажется предположить, что она определяется индивидуальными особенностями работодателя, разместившего вакансию $w_j$ а также количеством откликов на нее $n_j$:
    \begin{equation}
        \label{a_def}
        a_j = a(w_j, n_j)
    \end{equation}
    и если мы добавляем в нее показ рекомендации $w_r$, значение меняется следующим образом:

    \begin{multline}
        \label{last_sum}
        S = \sum_{i}^{k-1}\sum_{w_j \in Re_{i}} a(w_j, n_j)  * f(c_i, w_j)  - \\
        - \sum_{j : w_r \in Re_{j}} a(w_r, n_r)  * f(c_j, w_r) - \\
        + \sum_{j : w_r \in Re_{j}} a(w_r, n_r + 1)  * f(c_j, w_r)
        + a(w_r, n_r + 1)  * f(c_k, w_r)
    \end{multline}

    Т.e. если мы рекомендуем кандидату $c_k$ вакансию $w_r$ -- мы потенциально увеличиваем число откликов на нее, тем самым оказывая влияние на оклики всех остальных, откликнувшихся на $w_r$ и наша функция выигрыша требует пересчета для всех предыдущих откликов, а также для вновь добавленного. Таким образом, две вакансии стоит сравнивать между собой не просто по вероятности соответствия одного другому, но и по изменению функции выигрыша $a(w_r, n_r + 1)  * f(c_k, w_r)$, вызванному потенциальным откликом: т.e. насколько вероятно, что отклик вообще будет рассмотрен и какова вероятность получить приглашение.

    \section{Замечания и частные случаи}

    \begin{enumerate}
        \item Функция $f$ предполагается тут некоторой данностью - отдельным вопросом является ее нахождение, например ее можно искать в виде $f = p^{Re}_{i,j} * p^{I}_{j,i}$, где множители выражают вероятность отклика соискателем и приглашения работодателем соответсвенно
        \item В частности, если работодатель приглашет на интервью всех подряд, то $a_j = 1$, $b_j = 1$, если никого - $a_j = 0$, $b_j = 0$. Здесь и далее мы предполагаем $b_j$ некоторой константной ошибкой нанимающего, не зависящей от числа откликов.
        \item Пример (\ref{intro_example}) можно обобщить и на случай, когда только часть пересечения принадлежит какому-то одному множеству, а не все пересечение целиком
        \item Еще пара замечаний про общий вид (\ref{a_def}):
        \begin{enumerate}
            \item должна убывать по мере роста $n_j$ и должна быть 0 при каком-то довольно большом значении $n_j$: предполагается, что физически невозможно дать адекыватный ответ на 1000000+ откликов. Например, можно считать, что $a = \frac{\alpha_j}{n_j}$, где  $\alpha_j$ - некоторая константа, определяемая работодателем
            \item в общем случае зависит от числа \textit{релевантных откликов}, а не откликов вообще; в формулах мы данное замечание не учитываем, считая эти величины линейно зависимыми
        \end{enumerate}
        \item При достаточно больших значениях $n_j$ в (\ref{last_sum}) мы можем считать, что разница между треьим и вторым слагаемым мала, т.е. влиянием отклика на других кандидатов начиная с некоторого момента можно пренебречь
    \end{enumerate}

\end{document}